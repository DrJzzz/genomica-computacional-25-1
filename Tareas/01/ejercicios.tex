\documentclass[letter]{article}

\usepackage[spanish]{babel}
\usepackage[utf8]{inputenc}
\usepackage{amsmath}
\usepackage{graphicx}
\usepackage{hyperref}
\usepackage{tcolorbox}
\usepackage{algorithm2e}
\tcbuselibrary{theorems}
\usepackage{parskip}
\usepackage{mathtools}
\usepackage{amsmath}
\usepackage{color}
\usepackage{fancyvrb}
\usepackage{listings}
\usepackage{enumerate}
\usepackage{listings}
\graphicspath{ {images/} }
\usepackage{cite}
\lstset{language=Python}  

\newcommand{\commentedbox}[2]{%
  \mbox{
    \begin{tabular}[t]{@{}c@{}}
    $\boxed{\displaystyle#1}$\\
    #2
    \end{tabular}%
  }%
}
\usepackage{vmargin}
\usepackage[colorinlistoftodos]{todonotes}
\setmargins{1.5cm}       % margen izquierdo
{1.0cm}                  % margen superior
{18.0cm}                 % anchura del texto
{23.42cm}                % altura del texto
{10pt}                   % altura de los encabezados
{1cm}                    % espacio entre el texto y los encabezados
{0pt}                    % altura del pie de página
{2cm}

\title{Tarea 1}
\date{Fecha de entrega: Lunes 9 de septiembre del 2024 }
\author{Genómica Computacional - 2025-I}
\begin{document}
\maketitle

    \begin{quote}
      {\emph Durante mucho tiempo se creyó que esos libros impenetrables 
      correspondían a lenguas pretéritas o remotas. Es verdad que los hombres más
      antiguos, los primeros bibliotecarios, usaban un lenguaje asaz diferente del
      que hablamos ahora; es verdad que unas millas a la derecha la lengua es dialectal
      y que noventa pisos más arriba, es incomprensible.
      \hfill{La Biblioteca de Babel - JLB}}
    \end{quote}
    
    
    
    \section{Instrucciones}
    Responde las siguientes preguntas y haz las implementaciones en código 
    como se indica a continuación:\\
    \begin{itemize}
        \item Indica los integrantes de tu equipo en un archivo de texto llamado 00Integantes.txt.
    Todos deberán presentar la misma tarea dentro de la actividad en el classroom.\\
        \item Haz las implementaciones en un script python que se llame {\tt{tarea\_n.py}} o bien en un 
    notebook que se llame igual.\\
        \item Sube todos los archivos a la entrada de la tarea correspondiente en el classroom 
    (preferentemente sin comprimir a menos de que sea inevitable debido a su tamaño o complejidad).
    
    \end{itemize}
    
    \section{Expresiones regulares}
    
        \begin{enumerate}
            \item Considera la siguiente expresión regular, 
            $\tt{r'TATAG  [\^{}AT](T*|AC)TATA'}$,
            y señala las cadenas que contengan instancias de la misma.
            También indica las posiciones donde cada instancia se encuentra. 
            
           Ejemplo: En b) de la posición 10 a 20
           \begin{enumerate}
                \item TATACGCGTATAGAACTATAGCCCTATA
                \item TATAGCGTATAGGACTATAGTATA
                \item GTATGTATAGCCGACTTA
                \item TATAGCCGACTATA
            \end{enumerate}
           
            \item A continuación se presentan 10 secuencias hipotéticas.
            Diseña una expresión regular que detecte regiones codificantes válidas
            y específica cuales de las secuencias la cumplen:
        
        \begin{verbatim}
        1.  ATATATACATACTGGTAATGGGCGCGCGTGTGTTAAGTTCTGTTGTAGGGGTGATTAGGGGCG
        2.  GGCCCACACCCCACACCAATATATGTGGTGTGGGCTCCACTCTCTCGCGCTCGCGCTGGGGAT
        3.  ATAAGGTGTGTGGGCGCGCCCCGCGCGCGCGTTTTTTCGCGCGCCCCCGCGCGCGCGCGCGCG
        4.  GGCGCGGGACGCGGCGGCGGATCCCGATCCGTGCGTCAATACTATTATGGCCAGATAGAATAA
        5.  GTGCTGCTGCGGCGCCCACACCTATTATCTCTCTCTCTCTGCCTCTCCACCTCGGGGCTTAAT
        6.  GCGCTGCTGCTGGCTCGATGGGCGCGTGCGTCGTAGCTCGATGCTGGCTCGAGCTGTAATCTT
        7.  GGCGCTCGCTCGGATGCGCGGCCGGGCTCTCTGCTCGCGCTCGCTTCGCGCTCGTGACCGCTG
        8.  AATTGGTGCGCGCTCGCGCACACACAGAGAGAGGGTTTATATAGGATGATATATCCACATTGG
        9.  ATGCTGCTGCTGGCTCTGCTTGCGCTCTGCTCGCTGGGGTGTGTGTGCCGCGCGCTGCTGCTC
        10. GCTGGGCTCGCTCGATGCGCGCGGGCGCGCGACCGCGGACGGCGTCGCTGCTAAATGGGCTTC
        \end{verbatim}
    
    	Debes entregar una lista con el número de línea en las que 
    	hay una región codificante válida (como la hayas revisado en la clase
        de biología), es decir, si en las líneas 0, 1 y 5 hubiera 
    	entonces el resultado debería ser:

             $$ [0, 1, 5]$$
             
        \end{enumerate}
  
    \section{Probabilidad y estadística}

    \begin{enumerate}

        \item En el archivo ${\tt{promotores.txt}}$ se encuentra 
    		la lista de secuencias tomadas del genoma de {\it{Vitis vinifera}}
    		y cada una de las secuencias puede que tenga alguno de las 
    		diferentes formas en las que se ha encontrado el promotor GATA:
    
    		$\left \lbrace AGATAG, TGATAG, AGATAA, TGATAA  \right \rbrace$
    
    		Deseamos estudiar estas regiones en función del promotor GATA y 
    		por lo tanto lo primero que deseamos es saber cuántas veces aparecen 
    		los promotores en cada región.\\

            \begin{itemize}
                \item Haz un boxplot con la distribución de cada uno 
                de los promotores 

                \item ¿Cuál es la media y desviación estándar de cada
                promotor?
                
            \end{itemize}
    
        \item {\it Misterio} Haz un script en python que reciba como parámetro un entero 
        	que determinará la cantidad de iteraciones para el siguiente 
        	algoritmo
        
        	\begin{algorithm}[H]
        		\KwData{$M$ iteraciones}
        		\KwResult{Número flotante $x$}
        		$i \leftarrow 1$\;
        		$D \leftarrow 0$\;
        		\While{$i<M$}{
        			$i \leftarrow i + 1$\;
        			$x \leftarrow -1\leq uniform()\leq 1$\;
        			$y \leftarrow -1\leq uniform()\leq 1$\;
        			$d \leftarrow \sqrt{x^2 + y^2}$\;
        			\If{$d \leq 1$} {
        				$D \leftarrow D +1$\;
        			}
        		}
        
        		$x \leftarrow 4 \cdot D/i$\;
        		\Return x\;
        		\caption{Algoritmo misterio}
        		\label{alg:misterio}
        	\end{algorithm}
        	{\bf{Tip}:} Revisa la documentación del paquete {\tt random}, particularmente
        	de la biblioteca de {\tt numpy}.
        
        
        	¿Qué está calculando el algoritmo \ref{alg:misterio}?

        \item {\it Expansión - Modificación} Haz un función en python que reciba una cadena semilla y que con
        probabilidad {\tt p} mute alguna posición elegida con probabilidad uniforme de la cadena original 
        y con {\tt 1-p} la concatene con ella misma. 

        \[ f(\sigma) \leftarrow  \begin{dcases}            
                \operatorname{muta} (\sigma)   &  {\tt p} \\
                \sigma \sigma  &  {\tt 1-p}
        \end{dcases}
        \]

        Otra función debe recibir una cantidad de iteraciones para aplicar esta función.
        El alfabeto a usar es $\Sigma = \lbrace 0,1 \rbrace$

        Ejemplo:

        \begin{lstlisting}
        cadena=0
        cad_final = itera_n_veces(f(cadena(0.25)), 5)
        cad_final
        00100010
        \end{lstlisting}
    
        \item  Utilizando los siguientes datos de sensibilidad (93 por ciento)
        	y especificidad (99 por ciento) reportados para cierta prueba rápida 
        	de antígeno para detectar la infección por virus SARS-COV2 y considerando
        	una prevalencia actual de COVID en México estimada a partir del promedio 
        	de casos nuevos observados a lo largo de 2 semanas de 16000 casos activos
        	respecto a una población total de 120,000,000 de habitantes encuentra:
        	
        	\begin{enumerate}
        		\item ¿Cuál es la probabilidad de que si uno de ustedes se realiza
                    una prueba rápida de este tipo y ésta resulta positiva ustedes en 
                    realidad sean portadores del virus SARS-COV2?
                \item ¿Cuál es la probabilidad de que si la prueba resulta negativa
        			ustedes en realidad no sean portadores del virus SARS-COV2?
        		\item Entre marzo y junio de 2021 se tuvo un promedio de nuevos contagios
                    semanales de alrededor de 3000 casos, por lo que a lo largo de dos 
                    semanas se tendría una prevalencia aproximada de 6000 casos activos respecto
                    a 120000000 de habitantes. Calcula las probabilidades referidas en los 
                    dos incisos anteriores pero considerando este nuevo dato de prevalencia. 
                    ¿Qué puedes concluir respecto a las probabilidades obtenidas en ambos escenarios?, 
                    ¿consideras que en el caso de las pruebas de detección de COVID es necesaria 
                    una mayor sensibilidad o una mayor especificidad? Justifica tu respuesta.
              \end{enumerate}
        \item La estacionariedad es una característica propia de las series de tiempo. Una serie
        de tiempo es una sucesión de estados (valores) cuyo orden refleja un proceso cronológico. 
        La formación de una secuencia de ADN o ARN puede ser modelada dentro de este paradigma.
        Decimos que una serie o secuencia es estacionaria si cada uno de los estados presentes 
        en la misma provienen de una misma distribución de probabilidad. A continuación se te 
        presentarán 5 secuencias de ADN de hebra simple en las que tu tarea consistirá en listar 
        cuáles son estacionarias y cuáles no. Para lograr lo anterior se te recomienda:
        \begin{enumerate}
            \item Calcular la Esperanza y la Varianza de cada serie para distintas ventanas 
            "temporales".
            \item Recordar, qué tipo de variable aleatoria nos permite modelar la frecuencia de 
            los nucleótidos en una secuencia de ADN.
            \item Tu respuesta debe incluir, el código utilizado para responder la pregunta.
        \end{enumerate}   
        Secuencias:
        \begin{verbatim}
            0. CGGAGACTTTTCCACTGTCGTCGGAGTAGTAAAATAACGGTACGTCTTAGTGTGCACCATCGACTCTTT
            GTATTGCTCGTTAGGGGTCGCAGCCTCTTGTTAAGCCGTAATGGGTGATCCCCGCTCGTGAAACGGTGCGAT
            CCTGTGATCTGTCAGTATCGAAGGAGTGAAAAAGCGATTGCTAGCCGAGGCGTACCGTG
            
            1. CGGAGTCCCCCCCACCGCCGCCGGTGCTGCATATCTACGGCACGCCCCAGTGTGCACCATCGACTCTTT
            GTATTGCTCGTTAGGGGTCGCAGCCTCTTGTTAAGCCGTAATGGGTGATCCCCGCTCGTGAAACGGTGCGAT
            CCTGTGATCTATTGATGTTAGCAACATAGCCGCATAGTTATTGATTACAATATCTTATA
            
            2. CGGAGTCCCCCCCACCGCCGCCGGTGCTGCATATCTACGGCACGCCCCAGCGCGCTCCACCGACCCCCC
            GCACCGCCCGCCAGGGGCCGCAGCCCCCCGCCATGCCGCTACGGGCGTCCCCCGCCCGCGATTCGGCGCGAC
            CCCGCGACCCGCCTGCACCGTAGGAGCGTAATAGCGTCCGCCTGCCGAGGCGCACCGCG
            
            3. CCCTACCCGGCAGACCCCTCCACGCCCCCCCGTAGCACCGGAACCAGATCCGCGGAGCGGGGAGGGAGG
            CGGGGGGGGGAGGTCGGTCGGGGGTGGGGCGACAAGAGAGGAAACGGCAAGGGGAAGGGAGGAAAAGGAGTG
            GAACGGAGGATTCATAGTAACTTGCGAAACGCACACGAAATGTGAACCATCACTACCAG
        \end{verbatim}
    \end{enumerate}

Si quieren leer {\emph{La Biblioteca de Babel}} lo pueden hacer en \href {https://www.ingenieria.unam.mx/dcsyhfi/material_didactico/Literatura_Hispanoamericana_Contemporanea/Autores_B/BORGES/Babel.pdf}{esta liga}

\end{document}